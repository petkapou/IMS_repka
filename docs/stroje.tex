
\section{Zemědělské stroje}

\subsection{Traktor a jeho závěsná zařízení}
Výhodou traktoru je jeho univerzálnost. Zvolili jsme traktor John Deere 6210R\footnote{Test traktoru John Deere 6210R: \\\url{\detokenize{
https://www.fwi.co.uk/machinery/mid-range-tractor-test-john-deere-6210r
}},
\\\url{\detokenize{
http://www.danhel.cz/fotogalerie/predvadeci-john-deere-6210r-s-directdrive.html
}}.}.
\begin{itemize}
  \item Spotřeba 37,1 litrů/h.
  \item Dopravní rychlost 13,8-42 km/h.
\end{itemize}

\subsubsection{Secí stroj}
Zvolili jsme secí stroj AMAZONE CAYENA, který si sám kypří půdu\footnote{Technické údaje o stroji AMAZONE CAYENA: \\\url{\detokenize{
https://www.zavesnatechnika.cz/seci-stroj-cayena
}}.}.
\begin{itemize}
  \item Pracovní rychlost 8-15 km/h.
  \item Pracovní záběr 6 m.
\end{itemize}

\subsubsection{Postřikový stroj}
Zvolili jsme nesené postřikovače AMAZONE UF s čelní nádrží FT\footnote{Technické údaje o stroji AMAZONE UF: \\\url{\detokenize{
https://www.zavesnatechnika.cz/obrazky-soubory/uf-a958a.pdf?redir
}}.}.
\begin{itemize}
  \item Pracovní rychlost 10 km/h.
  \item Pracovní záběr 12-30 m.
\end{itemize}

\subsubsection{Rozmetadlo minerálních hnojiv}
Zvolili jsme stroj AMAZONE ZA-M Ultra Profis Hydro\footnote{Technické údaje o stroji AMAZONE ZA-M: \\\url{\detokenize{
https://www.zavesnatechnika.cz/obrazky-soubory/prospekt_amazone_zam-55bc0.pdf?redir
}}.}.
\begin{itemize}
  \item Pracovní rychlost 12 km/h.
  \item Pracovní šířka 30 m.
  \item Pracovní rychlost 27 ha/h.
\end{itemize}

\subsubsection{Rozmetadlo minerálních hnojiv}
Zvolili jsme návěs T730/3\footnote{Technické údaje - NÁVĚS, 12T, T730/3: \\\url{\detokenize{
http://www.agrotechnika.cz/zemedelska-technika/navesy-12t-t730-3/
}}.}  s nosností 12t.
\begin{itemize}
  \item Objem 15,3 $m^3$.
\end{itemize}

\subsection{Sklízecí mlátička}
Zvolili jsme sklízecí mlátičku New Holland CR10.90\footnote{Technické údaje o sklízecí mlátičce CR10.90: \\\url{\detokenize{
https://www.eagrotec.cz/potvrzeno-zapisem-do-guinessovy-knihy-rekordu-sklizeci-mlaticka-cr10.90-je-nejvykonnejsi-mlatickou-na-svete
}}.}. Mlátička má i zásobník, do něj lze ukládat zrna v dobu sklizně. Odpad zůstává na poli.
Traktor zatím jezdí mezi mlátičkou a skladem, mlátička za jízdy přesypává svůj obsah na vůz za traktorem.
\begin{itemize}
  \item Spotřeba 11,14 litrů/ha.
  \item Rychlost 5,9 km/h.
  \item Rychlost sklizně 10,03 ha/h.
  \item Šířka žacího ústrojí: 6,10-12,5 m.
  \item Objem zásobníku zrna 14500 l.
  \item Rychlost vyprázdnění 142 l/m.
\end{itemize}