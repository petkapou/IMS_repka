\section{2. úkol}
\subsection{Zadání}

Najděte asymptoty grafu funkce 

\begin{displaymath}
f(x) = x^2\Big(\frac{\pi}{4}-arctg\Big(\frac{x^2}{x^2-1}\Big)\Big)
\end{displaymath}

\subsection{Rozbor příkladu}

Máme najít asymptoty grafu funkce, což znamená najít svislé, šikmé i vodorovné asymptoty.

Svislou asymptotou rozumíme přímku ve tvaru $x=a$, jestliže 

\begin{displaymath}
\lim_{x \to a^-} f(x) = \pm \infty \bigvee\lim_{x \to a^+} f(x) = \pm \infty
\end{displaymath}

Vodorovnou asymptotou rozumíme přímku ve tvaru $y=a$, jestliže

\begin{displaymath}
\lim_{x \to +\infty} f(x) = c \bigvee\lim_{x \to - \infty} f(x) = c
\end{displaymath}

Šikmou asymptotou rozumíme přímku, ve tvaru $y=ax+b;a \neq 0$. Koeficienty $A$ a $B$ jsou definovány jako

\begin{displaymath}
A~= \lim_{x \rightarrow \infty}\Big( \frac{f(x)}{x}\Big) \quad\wedge\quad B = \lim_{a \rightarrow b}\big(\,f(x)-Ax\,\big)
\end{displaymath}

Tyto limity je třeba najít a ověřit.

\subsection{Řešení}

\subsubsection{Svislé asymptoty}
Funkce může mít svislou asymptotu pouze v~bodech, ve kterých je nespojitá. Určíme si proto body nespojitosti. Budou to body, ve kterých jmenovatel argumentu funkce $arctg(x) = 0$, protože nulou nelze dělit.

\begin{displaymath}
x^2-1 \neq 0 \implies x \neq 1 \wedge x \neq -1 \implies D(f) = \mathbb{R}
\end{displaymath}

Limita ze součinu dvou závorek je rovna $ \pm \infty$, pokud alespoň jedna z~těchto závorek je rovna $\pm \infty$. Obor hodnot funkce $arctg(x) = (-\pi/2,\pi/2)$. Můžeme tedy říct, že $\frac{\pi}{4}-arctg(\frac{x^2}{x^2-1})$ se nikdy nebude limitně blížit k~$\pm \infty$. Musel by se výraz $x^2$ rovnat v~bodech nespojitosti $\pm \infty$. Vzhledem k~tomu, že $(-1)^2 = 1^2 = 1$, můžeme tvrdit, že funkce nemá svisou asymptotu, protože v~bodech $\{-1,1\}$ je nespojitost prvního druhu. Ve výpočtu není třeba dále pokračovat.

\subsubsection{Vodorovné asymptoty}

\begin{displaymath}
\lim_{x \to - \infty}\underbrace{x^2}_{\rightarrow+\infty}\Big(\frac{\pi}{4}-\underbrace{arctg\Big(\frac{x^2}{x^2-1}\Big)}_{\rightarrow \pi/4}\Big) = \infty \cdot 0
\end{displaymath}
\begin{displaymath}
\lim_{x \to - \infty}\underbrace{x^2}_{\rightarrow+\infty}\Big(\frac{\pi}{4}-\underbrace{arctg\Big(\frac{x^2}{x^2-1}\Big)}_{\rightarrow \pi/4}\Big) = \infty \cdot 0
\end{displaymath}

Zjistíme, zda funkce je sudá.

\begin{displaymath}
x^2\Big(\frac{\pi}{4}-arctg\Big(\frac{x^2}{x^2-1}\Big)\Big) = 
(-x)^2\Big(\frac{\pi}{4}-arctg\Big(\frac{(-x)^2}{(-x)^2-1}\Big)\Big)
\end{displaymath}

Což platí. Funkce má tedy nanejvýše jednu vodorovnou asymptotu. Pro výpočet limity použijeme L'Hospitalovo pravidlo.

\begin{displaymath}
\lim_{x \to \pm \infty}x^2\Big(\frac{\pi}{4}-arctg\Big(\frac{x^2}{x^2-1}\Big)\Big) = \lim_{x \to \infty}\frac{\frac{\pi}{4}-arctg\Big(\frac{x^2}{x^2-1}\Big)}{x^{-2}} =
\end{displaymath}

\begin{displaymath}
= \lim_{x \to \pm \infty} \dfrac{\dfrac{1}{1+\Big(\dfrac{x^2}{x^2-1}\Big)^2}\cdot\dfrac{2x(x^2-1)-2x\cdot x^2}{(x^2-1)^2}}{-2x^{-3}} = \lim_{x \to \pm \infty}\dfrac{1}{1+\Big(\dfrac{x^2}{x^2-1}\Big)^2}\cdot\dfrac{2x(x^2-1)-2x\cdot x^2}{(x^2-1)^2} \cdot \Big(-\dfrac{1}{2}\Big) \cdot x^3
\end{displaymath}

\begin{displaymath}
= \lim_{x \to \pm \infty} \frac{-x^4+2x^2-1}{2x^4-2x^2+1} \cdot \frac{x^4}{x^4-2x^2+1} = \lim_{x \to \pm \infty} \frac {x^8\cdot(-1+ \overbrace{\cdots}^{\rightarrow0})}{x^8\cdot (2 +\underbrace{\cdots}_{\rightarrow0})} = -\frac{1}{2}
\end{displaymath}

Funkce má tedy jedinou vodorovnou asymtotu s~rovnicí
\begin{displaymath}
y = - \frac{1}{2}
\end{displaymath}

\subsubsection{Šikmé asymptoty}

Jelikož $\lim_{x \to \pm \infty} f(x) = c$, tak potom $\lim_{x \to \pm \infty} \frac{f(x)}{x} = 0$ Šikmá asymptota neexistuje. 
