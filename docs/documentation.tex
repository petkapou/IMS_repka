\documentclass[11pt,a4paper,titlepage]{article}
\usepackage[left=2cm,text={17cm,24cm},top=3cm]{geometry}
\usepackage[T1]{fontenc}
\usepackage[czech]{babel}
\usepackage[utf8]{inputenc}

\bibliographystyle{czplain}

%uvozovky
\newcommand{\ceskeuvozovky}[1]{\quotedblbase#1\textquotedblleft}
\begin{document}

\begin{titlepage}
\begin{center}
    {\LARGE\textsc{Vysoké učení technické v~Brně}}\\
    \smallskip
    {\Large\textsc{Fakulta informačních technologií}}\\
    \bigskip
    \vspace{\stretch{0.382}}
    \LARGE{Modelování a simulace - projekt}\\
    \smallskip
    \Huge{Produkce řepky v ČR}\\
    \vspace{\stretch{0.618}}
\end{center}
    {\Large Petr Kapoun - xkapou04 \\ Erik Kelemen - xkelem01 \hfill \today }
\end{titlepage}

\tableofcontents
\newpage

\section{Pojem informace}
Slovo informace pochází z~latinského in-formatio, což znamená utváření nebo ztvárnění. Je~to~velmi široký, mnohoznačný pojem, který se~užívá v~různých významech. V~nejobecnějším smyslu je~informace chápána jako údaj o~prostředí, jeho stavu a~procesech v~něm probíhajících. Informace snižuje nebo odstraňuje neurčitost systému.\cite{wiki_informace}
Pokud chceme tvořit texty, je~nutné si~ujasnit, jaké informace a~jak je chceme předávat.
\subsection{Informace podle Luciana Floridiho}
Od~roku 1990 Luciano Floridi prosazuje tzv.~„filozofii informace“ jako nezávislou oblast výzkumu. Podle něho význam slova „informace“ závisí na~tom, jakým způsobem je~informace definována. Zahrnuje čtyři vzájemně propojené fenomény tohoto slova:
\begin{itemize}
  \item Informace „o~něčem“ (např. jízdní řády).
  \item Informace „jako~něco“ (např. DNA, otisk prstu).
  \item Informace „pro~něco“ (např. algoritmus, instrukce).
  \item Informace „v~něčem“ (např. vzorce). \cite{Havac_informace}
\end{itemize}
\subsection{Informace podle Michaela Bucklanda}
Jeho vnímání informace záleží na~vztazích mezi znalostmi, objekty a~procesy:
\begin{itemize}
  \item Informace jako znalost - je~objektem informace jako procesu. Jedná se~o~nehmotnou entitu, nikoliv proces (znalosti, názory, víra).
  \item Informace jako dokument - hmatatelná entita, dokument. Jde o~informaci v~určité materiální reprezentaci (fixovanou).
  \item Informace jako proces - někdo někoho informuje, sděluje mu zprávu. Závisí na~kontextu. \cite{online_buckland}
\end{itemize}
\section{Hledání informací}
U~informací, které používáme ve~svých pracích, je~potřeba kontrolovat jejich správnost. V~mnoha případech je~samozřejmě situace méně jasná a~rozdílné názoty ilustrují, že~zde může být konfliktní pohled na~pravdu a~falešnost.\cite{Ar_Information}
\section{Textové sdílení informací}
V~moderní kultuře lidé méně mluví a~více píší . Nemám na~mysli rukopis, ale~textování a~psaní na~stroji.
\cite{Ar_Luv2TxT}
\textit{„Tištěná média mají dlouhou tradici, odvíjející se od~letáků, jednolistých prohlášení a~rozsáhle šířených krátkých polemik.“}\cite{Douda_informace} I~my dnes potřebujeme předat některé informace formou textu, příkladem může být elektronická pošta. U~některých rozsáhlejších textů, jako jsou manuály, bakalářské a~diplomové práce, je~nutné dodržovat strukturu textu a~další konvence. V~mnoha věcech nám pomůže znalost typografie. K~prezentaci vlastních myšlenek nám~může pomoci například sázecí prostředí. \LaTeX .
\section{Co~je~\LaTeX?}
Lze říci, že~prakticky neexistuje osobní počítač, na~němž by~nebyl k~dispozici textový editor nebo některý z~produktů kategorie DTP - Desk Top Publishing (publikování "na~stole"). \cite{RybickaLatex}
Latex je~generický sázecí systém, který používá tex jako svůj formátovací stroj.\cite{Latex_companion} Jedná se~o~velice elegantní sérii příkazů, které posílají naše myšlenky sázecímu systému, který poté vytvoří dokument pro~naše potěšení.\cite{programujte}
Díky své flexibilitě, lehkosti použití a~profesionální typografické kvalitě je~momentálně \LaTeX používán skoro ve~všech humanitních oblastech a~ve~vědě.\cite{Latex_companion}
\newpage
\bibliography{literatura}

\end{document}
